% This is the central and most important section of the report. Its objective must
% be to show, with linearity and clarity, the steps that have led to the
% definition of a decision model. The description of the working hypotheses,
% confirmed or denied, can be found in this section together with the description
% of the subsequent refining processes of the models. Comparisons between
% different models (e.g. heuristics vs. optimal models) in terms of quality of
% solutions, their explainability and execution times are welcome.

% Do not attempt to describe all the code in the system, and do not include large
% pieces of code in this section, use pseudo-code where necessary. Complete source
% code should be provided separately (in Appendixes, as separated material or as a
% link to an on-line repo). Instead pick out and describe just the pieces of code
% which, for example:

% \begin{itemize}
%     \item are especially critical to the operation of the system;
%     \item you feel might be of particular interest to the reader for some reason;
%     \item  illustrate a non-standard or innovative way of implementing an algorithm, data
%           structure, etc..
% \end{itemize}

% You should also mention any unforeseen problems you encountered when implementing the
% system and how and to what extent you overcame them. Common problems are:
% difficulties involving existing software.

% todo meglio metterla qui l'analisi del dataset? media, std,... usate dalla normale
\subsubsection{Training e valutazione}

\subsubsection{Dataset split}

Al fine di valutare i vari modelli più correttamente possibile il dataset è
stato suddiviso in \textit{training-set}, \textit{validation-set} e
\textit{test-set}. Quindi il training è stato usato per l'allenamento, il
validation per accertarsi che il modello non sia propenso all'overfitting ed il
test per la valutazione finale.

\subsubsection{Valutazione dell'errore}
% loss e metriche

Per quanto riguarda le performance in termini di errore la funzione di
\textit{loss} utilizzata in fase di training è il \textit{Root Mean Squared
Logaritmic Error (RMSLE)}, in quanto è la misura scelta dalla Kaggle challenge
per confrontare le performance dei vari partecipanti. Inoltre essa risulta
adeguata al problema considerando il vasto intervallo dei valori dei prezzi.

Di seguito la definizione e alcune osservazioni su di RMSLE insieme alle
ulteriori metriche utilizzate; ad esempio il \textit{Mean Absolute Error (MAE)},
che fornisce una più immediata comprensione rispetto al RMSLE. \\
Mean Absolute Error (MAE):
\begin{equation}
    \frac{1}{n} \sum_{i=1}^{n} | y_i - \hat{y_i} |
\end{equation}
\\
Mean Squared Error (MSE):
\begin{equation}
    \frac{1}{n} \sum_{i=1}^{n} ( y_i - \hat{y_i} )^2
\end{equation}
Il quadrato fornisce un peso maggiore agli errori dei valori più elevati\cite{rmse-or-mae}.
\\
Root Mean Squared Error (RMSE):
\begin{equation}
    \sqrt{ \frac{1}{n} \sum_{i=1}^{n} ( y_i - \hat{y_i} )^2}
\end{equation}
La radice dell'MSE, nella stessa scala dei valori.
\\
Mean Squared Logarithmic Error (MSLE):
\begin{equation}
    \frac{1}{n}
        \sum_{i=1}^{n}
            ( \log(y_i+1) - \log(\hat{y_i}+1) )^2
    =
    \frac{1}{n}
        \sum_{i=1}^{n}
            \log^2(\frac{y_i+1}{\hat{y_i}+1})
\end{equation}
Essendo calcolato a partire da un rapporto riflette l'errore relativo e di
conseguenza risulta efficace laddove i valore assoluti presentano variazioni considerevoli.
\\
Root Mean Squared Logarithmic Error (RMSLE):
\begin{equation}
    \sqrt{ 
        \frac{1}{n}
            \sum_{i=1}^{n}
                ( \log(y_i+1) - \log(\hat{y_i}+1) )^2
    }
\end{equation}
La radice dell'MSLE.

% dire da qualche parte qualche altra valutaione: e.g. la computazione richiesta
% dai vari approcci

\subsubsection{Valutazione dei costi computazionali}

Il numero dei parametri allenabili dei vari modelli sono stati annotati insieme
al tempo richiesto per l'esecuzione di 10 epoche di train sulla piattaforma
\textit{Google Colab} abilitando l'accelerazione hardware GPU.

\subsection{Categoriche}
\textbf{Dire qualcosa sulle categoriche}

Inizialmente sono state valutate le performance di regressione dei prezzi a
partire dalle sole features categoriche in modo tale da stabilire un punto di
partenza per la successiva analisi delle features testuali \textit{name} e
\textit{item\_description}, più complesse e computazionalmente costose.

Un semplice modello composto da due layer Densi è stato sufficiente a
raggiungere le seguenti performance alla fine della decima epoca con l'aiuto di
un layer Dropout interposto ad essi per contrastare l'overfitting.

% todo: qui mettere le performance

% parlare di quanto tempo richiedono le 10 epoche.
% del numero dei parametri del modello.
% di quanti neuroni hanno i layers e perchè (magari fare una prova con vari
% valori bassi e aumentarli senza esagerare fino a raggiungere le performance sopra)
% parlare di qualche altro parametro? di training del modello?

\subsection{Rappresentazione del Testo}

\subsubsection{Bag of Words}
Il modello bag-of-words è una rappresentazione semplificata  di testi, dove
ciascuno di essi è rappresentato come una "borsa" (vettore) delle sue parole.
Quindi crea un vettore per ogni documento contenete il conteggio delle
occorrenze di ciascuna parola del vocabolario nel documento. Questa tecnica
ignora la grammatica e persino l'ordine delle parole, ma mantiene la
molteplicità \cite{manning_raghavan_schutze_2008}.
\subsubsection{Tf-Idf}\label{section-tfidf}
Tf-Idf (Term frequency-Inverse document frequency) \cite{manning_raghavan_schutze_2008} risolve uno dei principali problemi che si hanno con Bag of Words, poiché quest'ultima tecnica non considera relazioni tra i vari documenti. Infatti, Tf-Idf \eqref{eq:tf-idf} considera l'importanza di una parola per un documento in una collezione di documenti (corpus).
Il vettore costruito per ogni documento tramite questa tecnica considera per ciascun termine del documento la frequenza con cui il termine compare in questo documento (Term Frequency, TF \eqref{eq:tf}) rispetto alla frequenza del termine in tutti i documenti (Inverse Document Frequency, IDF \eqref{eq:idf}).

\begin{equation}
\label{eq:tf-idf}
   Tf-Idf_{t,d} = tf_{t,d} \cdot idf_t
\end{equation}\\
\begin{equation}
\label{eq:tf}
   tf_{t,d} = \frac{n_{t,d}}{number of terms in the document d} 
\end{equation}\\
\begin{equation}
\label{eq:idf}
   idf_{t} = \log \frac{number of documents}{number of documents with term t} 
\end{equation}
Dove in \eqref{eq:tf-idf}\eqref{eq:tf}\eqref{eq:idf} t indica il termine e d indica il documento.
\\
\subsubsection{Word Embeddings}

Le precedenti tecniche sono semplici da realizzare, robuste e funzionali per molti tasks. Tuttavia queste tecniche semplici sono al limite in molti compiti perché trattano le parole come unità atomiche, senza alcuna correlazione poiché vengono rappresentate come indici di un vocabolario. 
Con lo sviluppo delle tecniche di Machine Learning sono nati modelli in grado di
apprendere rappresentazioni vettoriali di alta qualità di parole provenienti da
set di dati composti da miliardi di parole e con vocabolari di milioni di
parole.

Queste nuove tecniche prendono il nome di \textbf{word embeddings} e
consentono di rappresentare parole per mezzo di vettori densi e di lunghezza
fissa \cite{almeida2019word}, fornendo di conseguenza una rappresentazione più
efficiente rispetto alla \textit{Bag of words} sparsa e dimensionalmente più
costosa.

Inoltre questi vettori sono in grado di rispettare la similarità tra le parole
considerando più gradi di somiglianza e rappresentando la regolarità semantica e
sintattica, permettendo anche operazioni algebriche \cite{mikolov2013efficient}.


\subsubsection{GLOVE}
GLOVE \cite{pennington2014glove} è un algoritmo di apprendimento non supervisionato che fornisce una rappresentazione vettoriale di ogni parola, considerando le statistiche globali di ricorrenza parola per parola (word-word ??????) da un corpus e le rappresentazioni ottenute mostrano sottostrutture lineari dello spazio vettoriale delle parole. Inoltre, gli ideatori di questa tecnica hanno reso disponibile i word embeddings pre-addestrati.
Nel dettaglio in questo lavoro sono stati utilizzati i seguenti embeddings:
\begin{itemize}
	\item Wikipedia 2014 + Gigaword 5: allenato su 6 miliardi di parole e con un vocabolario di 400 mila parole;
	\item Common Crawl: allenato su 840 miliardi di parole e con un vocabolario di 2.2 milioni di parole (il più grande disponibile di GLOVE);
\end{itemize}

\subsection{modello}\label{Modello}
Nella realizzazione di questo progetto sono stati utilizzati due modelli principali che trattano le tecniche di rappresentazione utilizzate ed  entrambi i modelli condividono la seguente struttura:
\begin{itemize}
	\item \textbf{Concatenate layer}: concatena gli input in un singolo tensore;
	\item \textbf{Dropout layer}: per escludere una frazione dell'input e il fattore utilizzato è 0.2;
	\item \textbf{Dense layer}: composto da 32 neuroni e come funzione di attivazione usa la Relu;
	\item \textbf{Dropout layer}: utilizza un fattore di 0.2;
	\item \textbf{Dense layer}: composto da 16 neuroni e come funzione di attivazione usa la Relu;
	\item \textbf{Dense layer}: rappresenta il layer di output ed è composto da un neurone con funzione di attivazione lineare poiché il task è una regressione;
\end{itemize}
\subsubsection{Word Embeddins}
Il modello principale utilizzato per i Word Embeddings (sia pretrainati che non) tratta i seguenti input differentemente: descrizione del prodotto, variabili categoriche del prodotto (nome, categorie, brand, shipping e condizione) e nome del prodotto.
La descrizione e il nome vengono rappresentati sotto forma di interi, dove ogni intero rappresenta l'indice di un token in un dizionario.
Le variabili categoriche che non prevedevano un valore intero sono state convertite in valori interi.
La descrizione prevede un layer di embedding, il quale crea la rappresentazione vettoriale densa delle parole e nel caso di GLOVE i pesi utilizzati non sono apprendibili in quanto vengono utilizzati quelli pretrainati.
Successivamente sono stati utilizzati due layer LSTM Bidirezionali per trattare questo input e il primo layer prevede 16 celle LSTM e un fattore di dropout di 0.2, mentre il secondo prevede 8 celle LSTM e anch'esso utilizza un fattore di dropout di 0.2.
Infine sul campo descrizione viene utilizzato un GlobalMaxPooling1D per il sotto campionamento in una rappresentazione più compatta.
Anche il nome viene trattato tramite un layer di embedding (anche qui con GLOVE vengono utilizzati i pesi pretrainati), successivamente è stato utilizzato un layer di LSTM Bidirezionale con 12 celle e un fattore di dropout di 0.2; infine è stato utilizzato anche qui il GlobalMaxPooling1D.
Sulle variabili categoriche non sono state effettuate altre operazioni.
I tre nuovi input ottenuti vengono passati nel layer di concatenazione e quindi nella struttura spiegata precedentemente all'inizio della sezione \ref{Modello}.

Nel modello sono stati utilizzati layers di RNN per trattare i dati non
strutturati, poiché sono ottimi per l'elaborazione di dati sequenziali e nei
testi per assegnare un'interpretazione corretta la relazione delle parole è
molto importante. Infatti, le RNN elaborano una sequenza di input un elemento
alla volta, mantenendo in "memoria" informazioni sugli elementi passati della
sequenza \cite{liang2017text}.

\subsubsection{Transformers}
% diremo che il trasform supera lstm. todo pro-cons lstm: tipo memoria,...
Il Transformer è un'architettura proposta nel 2017 che si contrappone alle RNN
evitando quindi la ricorrenza e utilizzando esclusivamente un meccanismo di
\textit{attention} per rappresentare i rapporti di dipendenza di input e output.
Si basa su di una struttura \textit{encoder-decoder} dove l'encoder fornisce al
decoder una rappresentazione dell'input ed in seguito il decoder fornisce una
frase in output \cite{vaswani2017attention}.

% BERT’s model architecture is a multi-layer bidirectional Transformer encoder
% based on the original implementation

Una delle più note architetture basata sul concetto di Transformer è
\textbf{BERT}, ovvero \textit{Bidirectional Encoder Representations from
Transformers}; si tratta di un transformer multi-layer bidirezionale, cioè in
grado di apprendere relazioni di dipendenza di un dato elemento dell'input sia
rispetto agli input precedenti che ai successivi, che si basa esclusivamente su
di moduli encoder. È stato introdotto con il fine di fornire un modello
pre-allenato semplicemente adattabile ad un vasto range di applicazioni tramite
\textit{fine-tuning}. Tuttavia anche gli approcci \textit{feature-based} basati
su BERT risultano efficaci \cite{devlin2018bert}.

In questo progetto BERT è stato utilizzato con quest'ultimo approccio
feature-based occupando effettivamente nel modello la stessa posizione dedicata
agli embedding per entrambi i campi nome e descrizione; il resto del modello è
pressochè invariato rispetto a quello precedentemente descritto, ovvero a
seguito di BERT troviamo fino a due layer LSTM bidirezionali (uno per il nome),
un GlobalMaxPooling1D e gli ultimi due layer Densi entrambi preceduti da un
dropout.



