% This is the central and most important section of the report. Its objective must
% be to show, with linearity and clarity, the steps that have led to the
% definition of a decision model. The description of the working hypotheses,
% confirmed or denied, can be found in this section together with the description
% of the subsequent refining processes of the models. Comparisons between
% different models (e.g. heuristics vs. optimal models) in terms of quality of
% solutions, their explainability and execution times are welcome.

% Do not attempt to describe all the code in the system, and do not include large
% pieces of code in this section, use pseudo-code where necessary. Complete source
% code should be provided separately (in Appendixes, as separated material or as a
% link to an on-line repo). Instead pick out and describe just the pieces of code
% which, for example:

% \begin{itemize}
%     \item are especially critical to the operation of the system;
%     \item you feel might be of particular interest to the reader for some reason;
%     \item  illustrate a non-standard or innovative way of implementing an algorithm, data
%           structure, etc..
% \end{itemize}

% You should also mention any unforeseen problems you encountered when implementing the
% system and how and to what extent you overcame them. Common problems are:
% difficulties involving existing software.

% todo meglio metterla qui l'analisi del dataset? media, std,... usate dalla normale

\subsection{Training e valutazione}

\subsubsection{Dataset split}

Al fine di valutare i vari modelli più correttamente possibile il dataset è
stato suddiviso in \textit{training-set}, \textit{validation-set} e
\textit{test-set}. Quindi il training è stato usato per l'allenamento, il
validation per accertarsi che il modello non sia propenso all'overfitting ed il
test per la valutazione finale.

\subsubsection{Valutazione dell'errore}
% loss e metriche

Per quanto riguarda le performance in termini di errore la funzione di
\textit{loss} utilizzata in fase di training è il \textit{Root Mean Squared
Logaritmic Error (RMSLE)}, in quanto è la misura scelta dalla Kaggle challenge
per confrontare le performance dei vari partecipanti. Inoltre essa risulta
adeguata al problema considerando il vasto intervallo dei valori dei prezzi.

Di seguito la definizione e alcune osservazioni su di RMSLE insieme alle
ulteriori metriche utilizzate; ad esempio il \textit{Mean Absolute Error (MAE)},
che fornisce una più immediata comprensione rispetto al RMSLE. \\
Mean Absolute Error (MAE):
\begin{equation}
    \frac{1}{n} \sum_{i=1}^{n} | y_i - \hat{y_i} |
\end{equation}
\\
Mean Squared Error (MSE):
\begin{equation}
    \frac{1}{n} \sum_{i=1}^{n} ( y_i - \hat{y_i} )^2
\end{equation}
Il quadrato fornisce un peso maggiore agli errori dei valori più elevati\cite{rmse-or-mae}.
\\
Root Mean Squared Error (RMSE):
\begin{equation}
    \sqrt{ \frac{1}{n} \sum_{i=1}^{n} ( y_i - \hat{y_i} )^2}
\end{equation}
La radice dell'MSE, nella stessa scala dei valori.
\\
Mean Squared Logarithmic Error (MSLE):
\begin{equation}
    \frac{1}{n}
        \sum_{i=1}^{n}
            ( \log(y_i+1) - \log(\hat{y_i}+1) )^2
    =
    \frac{1}{n}
        \sum_{i=1}^{n}
            \log^2(\frac{y_i+1}{\hat{y_i}+1})
\end{equation}
Essendo calcolato a partire da un rapporto riflette l'errore relativo e di
conseguenza risulta efficace laddove i valore assoluti presentano variazioni considerevoli.
\\
Root Mean Squared Logarithmic Error (RMSLE):
\begin{equation}
    \sqrt{ 
        \frac{1}{n}
            \sum_{i=1}^{n}
                ( \log(y_i+1) - \log(\hat{y_i}+1) )^2
    }
\end{equation}
La radice dell'MSLE.

% dire da qualche parte qualche altra valutaione: e.g. la computazione richiesta
% dai vari approcci

\subsubsection{Valutazione dei costi computazionali}

Il numero dei parametri allenabili dei vari modelli sono stati annotati insieme
al tempo richiesto per l'esecuzione di 10 epoche di train sulla piattaforma
\textit{Google Colab} abilitando l'accelerazione hardware GPU.

\subsection{Categoriche}
\textbf{Dire qualcosa sulle categoriche}

Inizialmente sono state valutate le performance di regressione dei prezzi a
partire dalle sole features categoriche in modo tale da stabilire un punto di
partenza per la successiva analisi delle features testuali \textit{name} e
\textit{item\_description}, più complesse e computazionalmente costose.

Un semplice modello composto da due layer Densi è stato sufficiente a
raggiungere le seguenti performance alla decima epoca con l'aiuto di un layer
Dropout interposto ad essi per contrastare l'overfitting.

% todo: qui mettere le performance

% parlare di quanto tempo richiedono le 10 epoche.
% del numero dei parametri del modello.
% di quanti neuroni hanno i layers e perchè (magari fare una prova con vari
% valori bassi e aumentarli senza esagerare fino a raggiungere le performance sopra)
% parlare di qualche altro parametro? di training del modello?

\subsection{modello}
qualche spiegazione sulla definizione del modello.
perchè le lstm, cosa sono, paper, lstm per il testo.

performance gain delle lstm??
perchè non abbiamo provato le gru?

\subsection{Rappresentazione del Testo}
Descrizione dei vari approcci nella storia (quando sono stati creati)

deep learning e non

\subsubsection{clean o raw}
abbiamo valutato clean e raw; quali modelli beneficiano del cleanup, quali no?
perchè? il deep learning impara meglio se il testo è raw?

\begin{itemize}
    \item bag of word based models: count vectorizer - tf/idf
    \item embeddings: keras, glove pretrained. dire perchè non abbiamo provato
    qualcosa tipo word2vec?
    \item transformers con fine tuning
\end{itemize}

