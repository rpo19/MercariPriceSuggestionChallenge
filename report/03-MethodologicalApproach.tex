% This is the central and most important section of the report. Its objective must
% be to show, with linearity and clarity, the steps that have led to the
% definition of a decision model. The description of the working hypotheses,
% confirmed or denied, can be found in this section together with the description
% of the subsequent refining processes of the models. Comparisons between
% different models (e.g. heuristics vs. optimal models) in terms of quality of
% solutions, their explainability and execution times are welcome.

% Do not attempt to describe all the code in the system, and do not include large
% pieces of code in this section, use pseudo-code where necessary. Complete source
% code should be provided separately (in Appendixes, as separated material or as a
% link to an on-line repo). Instead pick out and describe just the pieces of code
% which, for example:

% \begin{itemize}
%     \item are especially critical to the operation of the system;
%     \item you feel might be of particular interest to the reader for some reason;
%     \item  illustrate a non-standard or innovative way of implementing an algorithm, data
%           structure, etc..
% \end{itemize}

% You should also mention any unforeseen problems you encountered when implementing the
% system and how and to what extent you overcame them. Common problems are:
% difficulties involving existing software.

% todo meglio metterla qui l'analisi del dataset? media, std,... usate dalla normale

\subsection{Valutazione}
come abbiamo valutato le performance: perchè abbiamo usato rmsle, cosa è. altre
misure di performance

\subsubsection{regressore random normale}
abbiamo tenuto come punto di riferimento un regressore random che fornisce
prezzi da una normale con media e varianza dei nostri prezzi.

\subsection{Categoriche}
\textbf{Dire qualcosa sulle categoriche}

Estratte performance di regressione a partire dalle sole categoriche.

Qualche spiegazione del modello scelto.

\subsection{modello}
qualche spiegazione sulla definizione del modello.
perchè le lstm, cosa sono, paper, lstm per il testo.

performance gain delle lstm??
perchè non abbiamo provato le gru?

\subsection{Rappresentazione del Testo}
Descrizione dei vari approcci nella storia (quando sono stati creati)

deep learning e non

\subsubsection{clean o raw}
abbiamo valutato clean e raw; quali modelli beneficiano del cleanup, quali no?
perchè? il deep learning impara meglio se il testo è raw?

\begin{itemize}
    \item bag of word based models: count vectorizer - tf/idf
    \item embeddings: keras, glove pretrained. dire perchè non abbiamo provato
    qualcosa tipo word2vec?
    \item transformers con fine tuning
\end{itemize}

