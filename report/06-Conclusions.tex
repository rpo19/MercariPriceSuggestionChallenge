% Conclusions should summarize the central points made in the Discussion section,
% reinforcing for the reader the value and implications of the work. If the
% results were not definitive, specific future work that may be needed can be
% (briefly) described. The conclusions should never contain ``surprises''.
% Therefore, any conclusions should be based on observations and data already
% discussed. It is considered extremely bad form to introduce new data in the
% conclusions.

Riassumendo dai risultati ottenuti si evince che il modello Keras permette di
ottenere ottime performance in cambio di costi computazionali accettabili,
perciò questa tecnica risulta consigliata nel caso in cui si abbia a
disposizione una piattaforma almeno comparabile con Google Colab.

Bag of Words è stata una piacevole sorpresa infatti pur essendo l'approccio più
semplice testato ottiene ottimi risultati ed è probabilmente attuabile anche
da chi ha particolari limitazioni computazionali.

È stato inoltre evidenziato che il pre-processing del testo effettuato
generalmente non giova alle performance e sarebbe interessante valutare approcci
alternativi a quello utilizzato nel progetto.

Nel caso si abbiano a disposizione grandi risorse computazionali sarebbe utile
valutare approfonditamente l'approccio basato su BERT che potrebbe portare a
performance ottimali.

Un altro approccio che varrebbe la pena testare sarebbe l'utilizzo di un embedding
pre-trainato con GloVe come punto di partenza per poi allenarne ulteriormente i
pesi magari ampliando le dimensioni dell'embedding in caso si abbiano risorse sufficienti.
