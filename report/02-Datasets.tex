% In this section the available data sets must be presented. The term dataset
% refers to any type of information source, for example web services for
% geolocation fall into this category. In addition, all necessary data
% manipulation processes, such as cleaning and enrichment with external sources,
% must be presented and discussed.

Il dati sono reperibili direttamente dalla pagina web della challenge; sono
presenti un dataset di train e uno test. È Per la valutazione è stato utilizzato
esclusivamente il train in quanto il test è pensato esclusivamente per la
challenge e non contenendo i prezzi target risulta inutile per la valutazione.

Il dataset di train rappresenta quindi il dataset preso in esame e ogni
riferimento successivo al dataset è da intendersi al dataset di train.

Il numero di prodotti contenuti è di circa 1.39 milioni e per ognuno di essi
sono fornite le seguenti caratteristiche.
% assicurarsi che il lettore sia in grado di capire che prodotti sono (già detto
% nell'intro?)

\subsubsection{Price}
\textbf{Price} rappresenta il prezzo di vendita dell'articolo, ovvero la
variabile target della regressione. Il suo valore medio è di circa \$26, con un
minimo pari a \$0, un massimo di \$2009 e una deviazione standard di circa \$38.
Analizzando meglio la distribuzione si nota che in generale i prezzi sono
relativamente bassi: inferiori a \$29 per il 75\% dei prodotti.

Dopo aver appreso dal sito ufficiale di Mercari che i prezzi impostabili sono
vincolati nell'intervallo [\$5,\$2000] sono stati rimossi tutti i prodotti con prezzo minori di 5 e maggiori di 2000.
% todo citare link \url{https://www.mercari.com/us/help_center/article/69}).
\subsubsection{Train id}
\textbf{Train\_id} rappresenta l'identificativo del prodotto nell'elenco.
\subsubsection{Name}
\textbf{Name} è il nome del prodotto sotto forma di dato non strutturato.
% todo dato non strutturato. meglio testo?
\subsubsection{Shipping}
\textbf{Shipping} rappresenta di chi, tra venditore e acquirente, sono a carico
le spese di spedizione: il valore 1 significa "a carico del
venditore", 0 invece dell'acquirente.

Il 45\% dei prodotti è spedito a carico del venditore (Shipping=1), mentre il
restante 55\% a carico dell'acquirente (Shipping=0).

Ci si potrebbe aspettare che i prodotti spediti a carico del venditore abbiano
prezzi più elevati; tuttavia, almeno per quanto riguarda l'intero dataset in
esame, è vero il contrario: il prezzo medio dei prodotti spediti a carico degli
acquirenti, circa \$30, è infatti superiore a quello dei prodotti restanti,
circa \$22.

% spese di spedizione sono incluse nel prezzo? todo

\subsubsection{Item condition}
\textbf{Item\_condition\_id} rappresenta lo stato del prodotto; questo valore
varia da 1 a 5. Il valore più frequente è 1, mentre 4 e 5 sono i più rari. La
Kaggle challenge non fornisce una descrizione dettagliata del significato. Si
può supporre che il valore 1, il più frequente, identifichi la condizione
migliore, mentre il valore 5 la condizione peggiore. Tuttavia, alla luce dei
prezzi medi per ogni condizione, la supposizione sembra essere errata: i
prodotti in condizione 5 hanno mediamente il prezzo maggiore, quelli in
condizione 4, invece, il prezzo minore; infine la differenza nel prezzo medio
degli articoli nelle condizioni 1,2 e 3 risulta molto lieve.
\subsubsection{Category Name}
% todo dire che abbiamo splittato in 3 le categorie
\textbf{Category\_name} rappresenta la categoria di prodotto a cui appartiene l'articolo. Nel dataset
sono presenti 1287 categorie univoche, ad esempio \textit{"Women/Tops \&
Blouses/T-Shirts"}, ed è facilmente osservabile una sorta di gerarchia di
sottocategorie dalla più generica alla più specifica.

Inoltre, circa 6000 articoli non hanno nessuna categoria assegnata, mentre
analizzando le categorie più numerose, si nota come l'abbigliamento
femminile sia molto popolare su Mercari: 5 categarie nelle top 10 sono infatti
di abbigliamento femminile. Trucco ed elettronica ricoprono anch'essi posizioni
rilevanti tra le categorie più quotate.

Siccome in più del 99\% dei casi il campo contiene 3 livelli di sottocategorie
il campo \textit{Category\_name} è stato diviso in 3 campi, uno per ogni livello,
assegnando la stringa "NA" a quegli articoli con uno o più livelli mancanti.
\subsubsection{Brand Name}
\textbf{Brand\_name} rappresenta il marchio dell'articolo; nel dataset sono
presenti 4809 brand differenti e più di 600 mila valori mancanti: poco meno
della metà dei prodotti totali; a questi prodotti è stato assegnato "NA" come \textit{Brand\_name}.
\subsubsection{Item Description}
% todo: scrivere quante parole hanno in media. serve poi nell'embedding? o più
% nel padding del text preprocessing 
\textbf{Item\_description} rappresenta la descrizione del prodotto sotto forma
di dato non strutturato.

Nel dataset sono presenti 4 istanze senza descrizione e
circa 82 mila con la stringa "no description yet"; in entrambi i casi
\textit{item\_description} è stato impostato a "NA".

Inoltre, non sembra esistere una relazione lineare tra lunghezza delle
descrizioni e prezzo, in quanto l'indice di correlazione di Pearson è prossimo a
zero; 0.048 per l'esattezza.
% todo, omettere? abbiamo calcolato la correlazione solo sulle descrizioni?

Analizzando le word cloud ottenute dai bigrammi delle descrizioni dopo aver
suddiviso i prodotti in quattro fasce di prezzo (figure \ref{fig:100},
\ref{Fig:50_100}, \ref{fig:30_50} e \ref{Fig:minore_30}) si riescono a notare
delle differenze sulle coppie parole più frequenti, identificabili perchè di
dimensioni più grandi.

Nella wordcloud dei prodotti con prezzo maggiore o uguale a \$100 (Figura
\ref{fig:100}) sono molto frequenti bigrammi che danno informazioni sulle buone
condizioni dei prodotti, ad esempio: 100 authentic, great condition e good
condition. Al diminuire del prezzo questi bigrammi diventano meno frequenti;
aumentano invece i bigrammi relativi a descrizioni mancanti.

\begin{figure}[H]
   \begin{minipage}{0.48\textwidth}
     \centering
     \includegraphics[width=.9\linewidth]{maggiore_100}
	\caption{Descrizioni dei prodotti con prezzo maggiore o uguale a \$100}
	\label{fig:100}   
	\end{minipage}\hfill
   \begin{minipage}{0.48\textwidth}
     \centering
     \includegraphics[width=.9\linewidth]{50_100}
     \caption{Descrizioni dei prodotti con prezzo tra \$50 e \$100}
     \label{Fig:50_100}
   \end{minipage}
\end{figure}

\begin{figure}[H]
   \begin{minipage}{0.48\textwidth}
     \centering
     \includegraphics[width=.9\linewidth]{30_50}
	\caption{Descrizioni dei prodotti con prezzo tra \$30 e \$50}
	\label{fig:30_50}   
	\end{minipage}\hfill
   \begin{minipage}{0.48\textwidth}
     \centering
     \includegraphics[width=.9\linewidth]{minore_30}
     \caption{Descrizioni dei prodotti con prezzo minore o uguale a \$30}
     \label{Fig:minore_30}
   \end{minipage}
\end{figure}
\subsubsection{Preprocessing del testo}
Inoltre, è stato effettuato un trattamento dei dati (cito paper sul text preprocessing) non strutturati convertendoli tutti in minuscolo).
Sui dati testuali è stata effettuata una fase di lemmatizzazione (perchè usa vocabolario su cui tagliare le parole e cito articolo). Successivamente, sulle nuove parole ottenute è stata effettuata una fase di pulizia di questi campi non strutturati eliminando le stopwords, la punteggiatura, tutti i caratteri di lunghezza pari a 1 che non sono numeri (è stato deciso di mantenere tutti i numeri poichè molte descrizioni senza di essi perdono di significato) e sono state eliminate le emoji.


I valori del campo category\_name sono stati codificati in interi tramite la tecnica label encoding.

% todo: coefficiente di correlazione di  pearson?
% todo: piccole analisi come correlazioni sono da mettere qui? oppure nella
% sezione dopo?
% todo: far trasparire che abbiamo valutato entrambi gli approcci di text
% cleaning o no
