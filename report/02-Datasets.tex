% In this section the available data sets must be presented. The term dataset
% refers to any type of information source, for example web services for
% geolocation fall into this category. In addition, all necessary data
% manipulation processes, such as cleaning and enrichment with external sources,
% must be presented and discussed.

I dati sono reperibili direttamente dalla pagina web della challenge; sono
presenti un dataset di train e uno di test. Per la valutazione è stato utilizzato
esclusivamente il train in quanto il test è pensato per la
challenge e non contenendo i prezzi target risulta inutile per la valutazione;
ogni riferimento successivo al dataset è quindi da intendersi al dataset di train.

Il numero di prodotti contenuti è di circa 1.39 milioni e per ognuno di essi
sono fornite le seguenti caratteristiche.
% assicurarsi che il lettore sia in grado di capire che prodotti sono (già detto
% nell'intro?)

\begin{itemize}

\itemsep0em 
\item \textbf{Price:} rappresenta il prezzo di vendita dell'articolo, ovvero la
variabile target della regressione. Sono stati considerato i prodotti con prezzo tra \$5 e
\$2000 in base ai vincoli definiti dall'e-commerce
\cite{mercari-how-to-set-a-price}; la maggior parte dei prodotti (il 75\%) ha un
prezzo inferiore ai \$29.

\item \textbf{Train\_id:} rappresenta l'identificativo del prodotto nell'elenco.

\item \textbf{Name:} è il nome del prodotto sotto forma di dato non strutturato.
% todo dato non strutturato. meglio testo?

\item \textbf{Shipping:} rappresenta chi, tra venditore e acquirente, si fa
carico delle spese di spedizione: 1 per "il venditore", 0 per l'acquirente.
Inaspettatamente il prezzo dei prodotti spediti a carico degli
acquirenti è mediamente superiore.

\item \textbf{Item\_condition\_id:} rappresenta lo stato del prodotto; questo valore
varia da 1 a 5. Il valore più frequente è 1, mentre 4 e 5 sono i più rari. La
Kaggle challenge non ne fornisce una descrizione dettagliata del significato.

\item \textbf{Category\_name:} rappresenta la categoria di prodotto a cui appartiene
l'articolo.
Circa 6000 articoli non hanno nessuna categoria assegnata, mentre i restanti
si suddividono in 1287 categorie, ad esempio \textit{"Women/Tops \&
Blouses/T-Shirts"}, dove è facilmente osservabile una gerarchia di
sottocategorie dalla più generica alla più specifica. Il campo
\textit{Category\_name} è stato diviso in 3 campi corrispondenti alle 3
sottocategorie principali assegnando la stringa "NA" a quegli articoli con uno o
più livelli mancanti.

\item \textbf{Brand\_name:} rappresenta il marchio dell'articolo; per più di 600
mila prodotti, ovvero poco meno della metà del totale, il campo risulta
mancante: a questi prodotti è stato assegnato "NA" come \textit{Brand\_name}.

% todo: scrivere quante parole hanno in media. serve poi nell'embedding? o più
% nel padding del text pre-processing 
\item \textbf{Item\_description:} rappresenta la descrizione del prodotto sotto forma
di dato non strutturato.
Nel dataset sono presenti 4 istanze senza descrizione e
circa 82 mila con la stringa "no description yet"; in entrambi i casi
\textit{item\_description} è stato impostato a "NA".

Nelle wordcloud ottenute dai bigrammi delle descrizioni (figure
\ref{fig:100} e \ref{Fig:minore_30}) si nota
che i prodotti con prezzo maggiore o uguale a \$100 (Figura \ref{fig:100}) sono
spesso descritti con informazioni che ne mostrano le buone condizioni, ad
esempio: "100 authentic", "great condition" e "good condition". Al diminuire del
prezzo queste coppie di parole diventano meno frequenti; aumentano invece quelle
relative a descrizioni mancanti.

\end{itemize}

\begin{figure}[H]
  \begin{minipage}{0.48\textwidth}
    \centering
    \includegraphics[height=2.2cm, keepaspectratio]{maggiore_100}
    \caption{Descrizioni dei prodotti con prezzo maggiore o uguale a \$100}
    \label{fig:100}
  \end{minipage}\hfill
  \begin{minipage}{0.48\textwidth}
    \centering
    \includegraphics[height=2.2cm, keepaspectratio]{minore_30}
    \caption{Descrizioni dei prodotti con prezzo minore o uguale a \$30}
    \label{Fig:minore_30}
  \end{minipage}
\end{figure}


\subsubsection{Preprocessing del testo}

Il pre-processing del testo è stato effettuato tramite le seguenti operazioni:
\begin{enumerate}
  \itemsep0em 
  \item conversione in caratteri minuscoli,
  \item lemmatizzazione,
  \item rimozione della punteggiatura,
  \item rimozione delle \textit{stopwords},
  \item rimozione delle parole di un solo carattere fatta eccezione per i
        numeri la cui rimozione spesso complica la comprensione della descrizione,
  \item rimozione delle emoji.
\end{enumerate}

In seguito si farà riferimento sia al pre-processing completo che al pre-processing
limitato, ovvero composto dalle sole operazioni "conversione in minuscolo" e
"rimozione di punteggiatura".

\subsubsection{Encoding}

Per quanto riguarda le variabili categoriche, ovvero \textit{Brand\_name} e le
sotto-categorie di \textit{Category\_name},
% todo è chiaro quali sono le
% categooriche? altrimenti chiarire
i valori sono stati codificati in interi tramite \textit{label encoding}.

I campi testuali invece sono stati codificati in forma vettoriale utilizzando
vari approcci che verranno in seguito discussi.

% tipologie: bow/tfidf, dizionario e poi embedding, tokenizer di bert

% todo: piccole analisi come correlazioni sono da mettere qui? oppure nella
% sezione dopo?
% todo: far trasparire che abbiamo valutato entrambi gli approcci di text
% cleaning o no --> messo in introduzione
