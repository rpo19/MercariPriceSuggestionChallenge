The discussion section aims at interpreting the results in light of the project's objectives. The most important goal of this section is to interpret the results so that the reader is informed of the insight or answers that the results provide. This section should also present an evaluation of the particular approach taken by the group. For example: Based on the results, how could the experimental procedure be improved? What additional, future work may be warranted? What recommendations can be drawn?
Il modello più performante è stato quello basato sull'embedding "custom" tramite i layers appositi e durante la valutazione dei layer RNN sono state valutate le seguenti tipologie: LSTM, GRU e
Bidirectional LSTM. Queste ultime, essendo in grado di catturare relazioni sia
con parole precedenti che con parole successive \cite{schuster1997bidirectional}, si sono dimostrati più performanti e
sono stati utilizzati nel modello finale.

Sono stati inoltre valutati layer convoluzionali a monte dei layer RNN; tuttavia
non avendo migliorato considerevolmente le performance sono stati omessi nel
modello finale.
% todo numeri dei layers?

\subsection{Transformers}

% todo: l'ho scritto nel 03. andrà modificato un po'.
% necessario aggiungere performance
Non è stato purtroppo possibile per motivazioni computazionali allenare questo
modello sulla totalità del dataset e le limitazioni delle risorse di Google
Colab hanno reso obbligata la scelta di utilizzare una piccola porzione del
dataset per il training. Tuttavia i risultati sul validation sono
particolarmente buoni considerando la ridotta quantità di esempi di training a
disposizione rispetto ai modelli precedenti.