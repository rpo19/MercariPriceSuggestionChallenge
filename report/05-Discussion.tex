% The discussion section aims at interpreting the results in light of the
% project's objectives. The most important goal of this section is to interpret
% the results so that the reader is informed of the insight or answers that the
% results provide. This section should also present an evaluation of the
% particular approach taken by the group. For example: Based on the results, how
% could the experimental procedure be improved? What additional, future work may
% be warranted? What recommendations can be drawn?
\subsection{Modello Finale}
\subsubsection{Parametri di training}

\paragraph{Numero di neuroni:} Il numero di neuroni del modello finale dei vari
layer è mostrato nelle figure \ref{fig:modeltemplate} e \ref{fig:kerasModel}
allo stesso modo del rate di dropout.

\paragraph{Optimizer:} Come optimizer è stato utilizzato Adam impostando il
learning rate a 1e-4, in quanto il training è risultato più stabile per la
maggior parte degli approcci ad eccezione dell'Embedding Keras che ha dimostrato
un comportamento migliore con il learning rate di default (1e-3).

\paragraph{Batchsize:} La batchsize è stata impostata a 256.

\paragraph{Regolarizzazione:} Oltre ai layer Dropout è stato testata la
regolarizzazione L2, tuttavia non ha portato a evidenti miglioramenti ed è
quindi stata omessa. Inoltre è stato impostato l'earlystopping a monitoraggio
dell'errore sul validation impostando pazienza a 3 epoche e il ripristino
automatico dei pesi più performanti.

\subsubsection{Layer aggiuntivi per il testo}

Per quanto concerne i layer aggiuntivi in coda agli Embedding sono state
valutate layer ricorsivi (RNN) in quanto ottimi per l'elaborazione di dati
sequenziali, come il testo, essendo in grado di mantenere in "memoria"
informazioni sugli elementi precedenti \cite{liang2017text}. Nello specifico
sono stati valutati LSTM, GRU e Bidirectional LSTM. Queste ultime, permettono di
catturare relazioni sia con parole precedenti che con parole successive
\cite{schuster1997bidirectional}, ed essendosi sono dimostrate più performanti
compaiono nel modello finale.

Sono stati inoltre valutati layer convoluzionali a precedere i layer RNN; tuttavia
non avendo migliorato considerevolmente le performance sono stati omessi nel
modello finale.

L'aggiunta di un layer GlobalMaxPooling1D a seguire le Bi\_LSTM (figura
\ref{fig:kerasModel}) si è dimostrata efficace sia per rendere la dimensionalità
compatibile con i successivi layer Densi che nell'effettuare un
\textit{downsampling} riducendo la dimensionalità insieme ai costi
computazionali.

\subsubsection{Dimensione Word Embedding}
La dimensione del word embedding Keras, ovvero quello allenato insieme al resto
del modello, è stata impostata a 50.

\subsection{Embedding pretrainati}
In tabella \ref{tab:restable} si nota come i modelli word embedding
pre-allenati (GloVe) richiedano un numero di epoche maggiore per convergere.

\subsection{Pulizia del testo}

Generalmente il preprocessing completo del testo non ha prodotto evidenti
miglioramenti sulle performance in termini di errore rispetto al preprocessing
limitato, nonostante nel caso degli embedding pre-trainati il preprossing
completo permette una maggiore copertura delle parole del vocabolario: da 32\%
a 38\% per GloVe6B e da 42\% a 52\% per GloVe840B.

Per quanto riguarda gli embedding ciò può essere dovuto al fatto che essendo in grado di catturare relazioni semantiche tra le parole non
necessitano di un alto grado di preprossesing che rimuovendo le stopwords, ad
esempio, potrebbe perfino complicare l'apprendimento del significato della
frase.

Risulta invece inaspettato il risultato ottenuto con il preprocessing totale
applicato a Bag of Word e Tf-Idf: l'errore è infatti leggermente più alto rispetto
a quello con preprocessing limitato. Si deduce che operazioni come
rimozione di stopwords e lemmatizzazione non hanno offerto un contributo
rilevante ma anzi un leggero peggioramento. 

Si nota invece una generale diminuzione del numero di parametri (figura
\ref{fig:resParam}) che in alcuni casi si traduce in una riduzione del tempo
richiesto da una singola epoca di train (figura \ref{fig:resEtime}).

\subsection{Performance}
Il modello più performante dal punto di vista dell'errore di regressione risulta
l'embedding Keras allenato con il resto del modello, evidenziato in verde nella
tabella \ref{tab:restable}. Seguono gli approcci Bag of Word e Tf-Idf;
quest'ultimo tuttavia non sembra migliorare i risulatati del più semplice BoW.

Dal punto di vista computazionale invece, la durata di una singola epoca di Bag
of Word è notevolmente minore rispetto a quella dell'embedding Keras allenato,
tuttavia considerando il numero di epoche necessarie alla convergenza ci
accorgiamo che i tempi di training risultano simili: 5000s per BoW e 5530s per
Keras. È necessario ricordare che però Keras è stato allenato con un learning
rate maggiore che per BoW risultava invece in una scarsa stabilità.

Sono invece gli Embedding GloVe pretrainati ad essere i più veloci nella singola
epoca di training, senza tener conto del modello di riferimento tenente conto
delle sole feature non testuali, avendo il numero minore di
parametri allenabili come mostrato in figura \ref{fig:resParam}. Anch'esse
tuttavia richiedono numerose epoche per convergere. Non risultano esserci
notevoli differenza ne di errori ne di costi computazionali tra i due embedding
pretrainati GloVe6B e GloVe840B.

\subsection{BERT}

Non è stato possibile per motivazioni computazionali allenare questo
modello sulla totalità del dataset ma è stata utilizzata porzione del
dataset per il training (circa 200 mila istanze) completando solo 5 epoche.
Tuttavia considerando queste limitazioni i risultati sembrano promettenti.

È stata una piacevole sorpresa il risultato ottenuto da Bag of Word che pur
essendo l'approccio più semplice testato ottiene ottimi risultati sia in termini
di errore che di richieste computazionali rappresentandone un buon compresso.