% The introduction should provide a clear statement of the problem posed by the
% project, and why the problem is of interest. It should reflect the scenario, if
% available. If needed, the introduction also needs to present background
% information so that the reader can understand the significance of the problem. A
% brief summary of the hypotheses and the approach your group used to solve the
% problem should be given, possibly also including a concise introduction to
% theory or concepts used later to analyze and to discuss the results.


% Descrizione della challenge originale: mercari price. Descrizione di cosa la
% challenge implica:
% - regressione a partire da features categoriche e testuali Dire che le
%   testuali sono importanti e più complicate (anche computazionalmente) e dire
%   che procediamo con un confronto dei diversi approcci esistenti per trattare
%   testi valutandoli in base alle performance ottenute e alla risorse
%   computazionali richieste però dei vari dati se ne parla nella sezione
%   dataset quindi forse è troppo presto
%   per parlare di testo


Il progetto trae origine dalla \textit{Kaggle challenge} \textbf{Mercari Price
Suggestion Challenge} \cite{mercari-price-suggestion-challenge} aperta a fine
Novembre 2017 che come viene reso chiaro dal sottotitolo \textit{"Can you
automatically suggest product prices to online sellers?"} pone l'obiettivo di
stimare il prezzo dei prodotti a partire da alcune loro caratteristiche. \\
Alla base di ciò vi è l'esigenza dell'e-commerce \textit{Mercari} \cite{mercari}
di offrire ai propri venditori un suggerimento sul prezzo di vendita dei
prodotti inseriti.

Si tratta quindi di un problema di \textit{regressione} che a partire dalle
varie caratteristiche dei prodotti, testuali e non, vuole calcolare il prezzo da
suggerire.
\\
Durante lo svolgimento del progetto si valuteranno vari approcci al problema,
sopratutto per quanto riguarda i dati di tipo testuale, sia in termini di errore
rispetto ai dati di \textit{train} che di costi computazionali.

In particolare saranno valutati i seguenti modelli di rappresentazione del
testo: Bag of Words, Tf-Idf, Word Embedding, Word Embedding pre-allenato (GloVe),
Feature extraction con Transfomer pre-allenato (DistilBert); verrà inoltre
valutata l'efficacia del pre-processing del testo.

