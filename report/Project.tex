%%%%%%%%%%%%%%%%%%%%%%%%%%%%%%%%%%%%%%%%%
% University Assignment Title Page 
% LaTeX Template
% Version 1.0 (27/12/12)
%
% This template has been downloaded from:
% http://www.LaTeXTemplates.com
%
% Original author:
% WikiBooks (http://en.wikibooks.org/wiki/LaTeX/Title_Creation)
%
% License:
% CC BY-NC-SA 3.0 (http://creativecommons.org/licenses/by-nc-sa/3.0/)
% 
% Instructions for using this template:
% This title page is capable of being compiled as is. This is not useful for 
% including it in another document. To do this, you have two options: 
%
% 1) Copy/paste everything between \begin{document} and \end{document} 
% starting at \begin{titlepage} and paste this into another LaTeX file where you 
% want your title page.
% OR
% 2) Remove everything outside the \begin{titlepage} and \end{titlepage} and 
% move this file to the same directory as the LaTeX file you wish to add it to. 
% Then add \input{./title_page_1.tex} to your LaTeX file where you want your
% title page.
%
%%%%%%%%%%%%%%%%%%%%%%%%%%%%%%%%%%%%%%%%%
%\title{Title page with logo}
%----------------------------------------------------------------------------------------
%	PACKAGES AND OTHER DOCUMENT CONFIGURATIONS
%----------------------------------------------------------------------------------------

\documentclass[12pt]{article}
\usepackage[italian]{babel}
\usepackage[utf8x]{inputenc}
%\usepackage{amsmath}
\usepackage{graphicx}
%\usepackage[colorinlistoftodos]{todonotes}
\usepackage{float}

\usepackage[hidelinks]{hyperref}
\usepackage[all]{hypcap}

\usepackage[table]{xcolor}

\begin{document}
\pagenumbering{roman}

\begin{titlepage}

\newcommand{\HRule}{\rule{\linewidth}{0.5mm}} % Defines a new command for the horizontal lines, change thickness here

\center % Center everything on the page
 
%----------------------------------------------------------------------------------------
%	HEADING SECTIONS
%----------------------------------------------------------------------------------------

\textsc{\LARGE Università degli studi di Milano-Bicocca}\\[1cm] % Name of your university/college
\textsc{\Large Advanced Machine Learning }\\[0.3cm] % Major heading such as course name
\textsc{\large Final Project}\\[0.1cm] % Minor heading such as course title

%----------------------------------------------------------------------------------------
%	TITLE SECTION
%----------------------------------------------------------------------------------------

\HRule \\[0.4cm]
{ \huge \bfseries Mercari Price Suggestion Challenge}\\[0.4cm] % Title of your document
\HRule \\[1.5cm]
 
%----------------------------------------------------------------------------------------
%	AUTHOR SECTION
%----------------------------------------------------------------------------------------

\large
\emph{Authors:}\\
Gabriele Ferrario -  817518 - g.ferrario@campus.unimib.it \\   % Your name
Riccardo Pozzi - 807857 - r.pozzi@campus.unimib.it   \\[1cm] % Your name

% If you don't want a supervisor, uncomment the two lines below and remove the section above
%\Large \emph{Author:}\\
%John \textsc{Smith}\\[3cm] % Your name

%----------------------------------------------------------------------------------------
%	DATE SECTION
%----------------------------------------------------------------------------------------

{\large \today}\\[2cm] % Date, change the \today to a set date if you want to be precise

%----------------------------------------------------------------------------------------
%	LOGO SECTION
%----------------------------------------------------------------------------------------

\includegraphics{logo.png}\\[1cm] % Include a department/university logo - this will require the graphicx package
 
%----------------------------------------------------------------------------------------

\vfill % Fill the rest of the page with whitespace

\end{titlepage}

\pagenumbering{arabic}
\begin{abstract}
    % The ABSTRACT is not a part of the body of the report itself. Rather, the
% abstract is a brief summary of the report contents that is often separately
% circulated so potential readers can decide whether to read the report. The
% abstract should very concisely summarize the whole report: why it was written,
% what was discovered or developed, and what is claimed to be the significance of
% the effort. The abstract does not include figures or tables, and only the most
% significant numerical values or results should be given.

Il progetto si incentra su di un problema di regressione derivato dalla Kaggle
Challenge Mercari Price Suggestion che consiste nella predizione del prezzo di
vendita dei prodotti a partire da feature testuali e non. È stata effettuata una
valutazione sia in termini di performance di regressione che di costi
computazionali di diversi approcci al problema, soprattutto relativi alla
rappresentazione del testo, tra cui Bag of Words e Word Embedding. È stato
valutato il contributo del pre-processing del testo e l'efficacia di diversi
tipi di layer neurali nella gestione del testo. In conclusione sono state
ottenute ottime performance utilizzando un Word Embedding allenabile tramite il
layer Embedding di Keras in combinazione a layer LSTM bidirezionali. Bag of
Words è risultata una buona alternativa pur essendo più semplice e
computazionalmente più economica. Per quanto riguarda il pre-processing del
testo, l'approccio condotto non ha portato miglioramenti rilevanti.
\end{abstract}

\section{Introduction}
% The introduction should provide a clear statement of the problem posed by the
% project, and why the problem is of interest. It should reflect the scenario, if
% available. If needed, the introduction also needs to present background
% information so that the reader can understand the significance of the problem. A
% brief summary of the hypotheses and the approach your group used to solve the
% problem should be given, possibly also including a concise introduction to
% theory or concepts used later to analyze and to discuss the results.


% Descrizione della challenge originale: mercari price. Descrizione di cosa la
% challenge implica:
% - regressione a partire da features categoriche e testuali Dire che le
%   testuali sono importanti e più complicate (anche computazionalmente) e dire
%   che procediamo con un confronto dei diversi approcci esistenti per trattare
%   testi valutandoli in base alle performance ottenute e alla risorse
%   computazionali richieste però dei vari dati se ne parla nella sezione
%   dataset quindi forse è troppo presto
%   per parlare di testo


Il progetto trae origine dalla \textit{Kaggle challenge} \textbf{Mercari Price
Suggestion Challenge} \cite{mercari-price-suggestion-challenge} aperta a fine
Novembre 2017 che come viene reso chiaro dal sottotitolo \textit{"Can you
automatically suggest product prices to online sellers?"} pone l'obiettivo di
stimare il prezzo dei prodotti a partire da alcune loro caratteristiche. \\
Alla base di ciò vi è l'esigenza dell'e-commerce \textit{Mercari} \cite{mercari}
di offrire ai propri venditori un suggerimento sul prezzo di vendita dei
prodotti inseriti.

Si tratta quindi di un problema di \textit{regressione} che a partire dalle
varie caratteristiche dei prodotti, testuali e non, vuole calcolare il prezzo da
suggerire.
\\
Durante lo svolgimento del progetto si valuteranno vari approcci al problema,
sopratutto per quanto riguarda i dati di tipo testuale, sia in termini di errore
rispetto ai dati di \textit{train} che di costi computazionali.

In particolare saranno valutati i seguenti modelli di rappresentazione del
testo: Bag of Words, Tf-Idf, Word Embedding, Word Embedding pre-allenato (GloVe),
Feature extraction con Transfomer pre-allenato (DistilBert); verrà inoltre
valutata l'efficacia del pre-processing del testo.



\section{Datasets}
In this section the available data sets must be presented. The term dataset refers to any type of information source, for example web services for geolocation fall into this category. 
In addition, all necessary data manipulation processes, such as cleaning and enrichment with external sources, must be presented and discussed.



Il Dataset consiste in un elenco di 1391082 prodotti descritti tramite le seguenti caratteristiche:
\subsubsection{Price}
textbf{Price} rappresenta il prezzo per il quale l'articolo è stato venduto (variabile target).
Il prezzo medio è di circa \$26, con un valore minimo pari a \$0 e un valore massimo pari a \$2009; inoltre, presenta una deviazione standard di circa \$38.
Analizzando i percentili ci si accorge che i prezzi sono relativamente bassi in quanto il 75\% dei prodotti hanno un prezzo al di sotto di \$29.
\subsubsection{Shipping}
\textbf{Shipping} è caratterizzato dal valore 1 se la tassa di spedizione è a carico del venditore, altrimenti 0 se è a carico dell'acquirente.
Questo attributo è decentemente ripartito tra i venditori e gli acquirenti, in quanto il 55\% dei prodotti prevede un valore di 0.
Analizzando i prezzo degli articoli ci si aspetta che per quelli che la tassa di spedizione è a carico del venditore avranno un prezzo più alto. Tuttavia, ci sono una serie di fattori contrastanti. Questo può essere vero all'interno di specifiche categorie di prodotti e condizioni degli articoli, ma non quando si confrontano gli articoli sul totale. 
Infatti, il prezzo medio pagato dagli utenti che devono pagare le spese di spedizione (circa \$30) è superiore a quelli che non richiedono costi di spedizione aggiuntivi (circa \$22).
\subsubsection{Category Name}
\textbf{Category\_name} rappresenta la categoria di prodotto a cui appartiene l'articolo.
Nel dataset sono presenti 1287 categorie univoche e tra ognuna di esse si vede una categoria principale/generale, seguita da due o più sottocategorie più specifiche (ad esempio: Women/Tops \& Blouses/T-Shirts). Inoltre, ci sono 6327 articoli che non hanno una categoria assegnata.
Infine, analizzando le dieci categorie più popolari, si nota che l'abbigliamento femminile è molto popolare su Mercari. Infatti, di queste prime dieci categorie 5 sono di abbigliamento femminile; Anche il trucco e l'elettronica sono categorie molto quotate.
\subsubsection{Brand Name}
\textbf{Brand\_name} rappresenta il marchio dell'articolo; nel dataset sono presenti 4809 valori differenti e 632682 valori mancanti.
\subsubsection{Item Description}
textbf{Item\_description} rappresenta la descrizione del prodotto sotto forma di dato non strutturato. Nel dataset sono presenti 4 istanze senza descrizione e 82494 descrizioni con la stringa "no description yet".
Inoltre, non esiste una correlazione tra la lunghezza delle descrizioni e il prezzo, in quanto c'è una correlazione di 0.048.
\textbf{AGGIUNGERE WORDCLOUD}

\section{The Methodological Approach}
% This is the central and most important section of the report. Its objective must
% be to show, with linearity and clarity, the steps that have led to the
% definition of a decision model. The description of the working hypotheses,
% confirmed or denied, can be found in this section together with the description
% of the subsequent refining processes of the models. Comparisons between
% different models (e.g. heuristics vs. optimal models) in terms of quality of
% solutions, their explainability and execution times are welcome.

% Do not attempt to describe all the code in the system, and do not include large
% pieces of code in this section, use pseudo-code where necessary. Complete source
% code should be provided separately (in Appendixes, as separated material or as a
% link to an on-line repo). Instead pick out and describe just the pieces of code
% which, for example:

% \begin{itemize}
%     \item are especially critical to the operation of the system;
%     \item you feel might be of particular interest to the reader for some reason;
%     \item  illustrate a non-standard or innovative way of implementing an algorithm, data
%           structure, etc..
% \end{itemize}

% You should also mention any unforeseen problems you encountered when implementing the
% system and how and to what extent you overcame them. Common problems are:
% difficulties involving existing software.

% todo meglio metterla qui l'analisi del dataset? media, std,... usate dalla normale
\subsubsection{Training e valutazione}

\subsubsection{Dataset split}

Al fine di valutare i vari modelli più correttamente possibile il dataset è
stato suddiviso in \textit{training-set}, \textit{validation-set} e
\textit{test-set}. Quindi il training è stato usato per l'allenamento, il
validation per accertarsi che il modello non sia propenso all'overfitting ed il
test per la valutazione finale.

\subsection{Categoriche}
\textbf{Dire qualcosa sulle categoriche}

Inizialmente sono state valutate le performance di regressione dei prezzi a
partire dalle sole features categoriche in modo tale da stabilire un punto di
partenza per la successiva analisi delle features testuali, \textit{name} e
\textit{item\_description}, più complesse e computazionalmente costose.

È stato utilizzato un semplice modello composto da due layer Densi con un layer
Dropout interposto per contrastare l'overfitting.
Questo modello verrà in seguito ampliato per utilizzare anche le feature
testuali a seconda dei vari approcci.
La figura \ref{fig:modeltemplate} ne mostra un template.

\begin{figure}[h!]
	\centering
	\includegraphics[
		height=8cm,
		keepaspectratio,
	]{modelTemplate.png}
	\caption{Template del modello}
	\label{fig:modeltemplate}
\end{figure}

% todo: mettere le performance nella sezione dopo
% todo: immagine del modello

% parlare di quanto tempo richiedono le 10 epoche.
% del numero dei parametri del modello.
% di quanti neuroni hanno i layers e perchè (magari fare una prova con vari
% valori bassi e aumentarli senza esagerare fino a raggiungere le performance sopra)
% parlare di qualche altro parametro? di training del modello?

\subsection{Rappresentazione del Testo}
% todo parlare della dimensionalità. di cosa è un dizionario

\subsubsection{Bag of Words}
% todo: dire qualcosa della dimensionalità di bow e tf idf?
Il modello bag-of-words è una semplice rappresentazione di testi, dove ciascuno
di essi è rappresentato come una "borsa" di parole; viene creato un vettore per
ogni documento che contiene il numero di occorrenze di ciascuna parola del
vocabolario nel documento. Questa tecnica ignora la grammatica e persino
l'ordine delle parole \cite{manning_raghavan_schutze_2008}.

\subsubsection{Tf-Idf}\label{section-tfidf} Tf-Idf (Term frequency-Inverse
document frequency) \cite{manning_raghavan_schutze_2008}, al contrario di Bag of
Words, considera l'importanza di una parola rispetto ad un documento alla luce
dell'importanza della stessa nell'intera collezione di documenti utilizzando la
frequenza in cui il termine compare prime nel documento e poi nella collezione
come mostrano le formule seguenti.% (corpus). ?
% todo: va bene parlare di documenti? vogliamo addattare più al nostro dataset?

\begin{equation}
\label{eq:tf-idf}
   Tf-Idf_{t,d} = tf_{t,d} \cdot idf_t
\end{equation}
\\
\begin{equation}
\label{eq:tf}
   tf_{t,d} = \frac{n_{t,d}}{|d|} 
\end{equation}
\\
\begin{equation}
\label{eq:idf}
   idf_{t} = \log \frac{|D|}{|\{d: t \in d\}|} 
\end{equation}
Dove in \eqref{eq:tf-idf}\eqref{eq:tf}\eqref{eq:idf} t indica il termine, d
indica il documento, D l'insieme dei documenti e le cardinalità di d e D si
riferiscono rispettivamente al numero di termini e a quello di documenti.



Per quanto riguarda il modello è stata sufficiente l'aggiunta un layer di
concatenazione in testa così da renderlo capace di accettare anche gli input
testuali nella forma di BoW.

% abbiamo dovuto aumentare i parametri rispetto alle categoriche?

\subsubsection{Word Embeddings}

Le precedenti tecniche sono semplici da realizzare, robuste e funzionali per
numerosi task. Tuttavia sono poco performanti in alcune applicazioni poiché
trattano le parole come unità atomiche, senza apprenderne relazioni complesse
ad esempio di similarità o causalità.
% nel senso che non si capisce se una parola è simile ad un altra e non si hanno
% relazioni del tipo una parola segue sempre l'altra. todo: è giusto dire
% causalità?

Con lo sviluppo delle tecniche di Machine Learning si sono diffusi i cosiddetti
\textit{word embedding} che consentono di rappresentare parole per mezzo di
vettori densi e di lunghezza fissa \cite{almeida2019word}, fornendo di
conseguenza una rappresentazione più efficiente rispetto alla \textit{Bag of
words} sparsa e dimensionalmente più costosa.
% todo: meglio dire prima che la bow è sparsa ?

Inoltre questi vettori sono in grado di apprendere la similarità tra parole così
come regole semantiche o sintattiche, permettendo anche operazioni algebriche
\cite{mikolov2013efficient}.
% dobbiamo sapere cosa significa semantico e sintattico

% word embedding di keras
\subsubsection{Keras Embedding Layer}

Keras fornisce un implementazione di embedding sotto la forma di layer neurale
e ne rende quindi possibile l'addestramento insieme al resto del modello;
altrimenti permette di fissare il valore dei pesi semplificando l'integrazione
con modelli pre-allenati.

% Nel modello è stata utilizzata l'implementazione di word embedding fornita da
% Keras sotto forma di layer. La sua dimensionalità è stata impostata a 50 e to
% be continued... todo dire ciò in qualche modo


\subsubsection{GLOVE}
GLOVE (Global Vectors) è un modello di apprendimento non supervisionato allenato
a partire dalle statistiche globali di co-occorrenza delle varie parole
\cite{pennington2014glove}. Sono inoltre disponibili modelli GLOVE
pre-addestrati; seguono quelli utilizzati nel progetto.
\begin{itemize}
	\item Wikipedia 2014 + Gigaword 5: allenato su 6 miliardi di parole e con un
	vocabolario di 400 mila parole;
	\item Common Crawl: allenato su 840 miliardi di parole e con un vocabolario
	di 2.2 milioni di parole (il più grande tra i modelli GLOVE pre-addrestrati);
\end{itemize}

\subsection{modello}\label{Modello}
Nella realizzazione di questo progetto sono stati utilizzati due modelli
principali che trattano le tecniche di rappresentazione utilizzate ed  entrambi
i modelli condividono la seguente struttura:
\begin{itemize}
	\item \textbf{Concatenate layer}: concatena gli input in un singolo tensore;
	\item \textbf{Dropout layer}: per escludere una frazione dell'input e il fattore utilizzato è 0.2;
	\item \textbf{Dense layer}: composto da 32 neuroni e come funzione di attivazione usa la Relu;
	\item \textbf{Dropout layer}: utilizza un fattore di 0.2;
	\item \textbf{Dense layer}: composto da 16 neuroni e come funzione di attivazione usa la Relu;
	\item \textbf{Dense layer}: rappresenta il layer di output ed è composto da un neurone con funzione di attivazione lineare poiché il task è una regressione;
\end{itemize}
\subsubsection{Word Embeddins}
Il modello principale utilizzato per i Word Embeddings (sia pretrainati che non)
tratta i seguenti input differentemente: descrizione del prodotto, variabili
categoriche del prodotto (nome, categorie, brand, shipping e condizione) e nome
del prodotto.
La descrizione e il nome vengono rappresentati sotto forma di interi, dove ogni
intero rappresenta l'indice di un token in un dizionario.
Le variabili categoriche che non prevedevano un valore intero sono state
convertite in valori interi.
La descrizione prevede un layer di embedding, il quale crea la rappresentazione
vettoriale densa delle parole e nel caso di GLOVE i pesi utilizzati non sono
apprendibili in quanto vengono utilizzati quelli pretrainati.
Successivamente sono stati utilizzati due layer LSTM Bidirezionali per trattare
questo input e il primo layer prevede 16 celle LSTM e un fattore di dropout di
0.2, mentre il secondo prevede 8 celle LSTM e anch'esso utilizza un fattore di
dropout di 0.2.
Infine sul campo descrizione viene utilizzato un GlobalMaxPooling1D per il sotto
campionamento in una rappresentazione più compatta.
Anche il nome viene trattato tramite un layer di embedding (anche qui con GLOVE
vengono utilizzati i pesi pretrainati), successivamente è stato utilizzato un
layer di LSTM Bidirezionale con 12 celle e un fattore di dropout di 0.2; infine
è stato utilizzato anche qui il GlobalMaxPooling1D.
Sulle variabili categoriche non sono state effettuate altre operazioni.
I tre nuovi input ottenuti vengono passati nel layer di concatenazione e quindi
nella struttura spiegata precedentemente all'inizio della sezione \ref{Modello}.

Nel modello sono stati utilizzati layers di RNN per trattare i dati non
strutturati, poiché sono ottimi per l'elaborazione di dati sequenziali e nei
testi per assegnare un'interpretazione corretta la relazione delle parole è
molto importante. Infatti, le RNN elaborano una sequenza di input un elemento
alla volta, mantenendo in "memoria" informazioni sugli elementi passati della
sequenza \cite{liang2017text}.

\subsubsection{Transformers}
% diremo che il trasform supera lstm. todo pro-cons lstm: tipo memoria,...
Il Transformer è un'architettura proposta nel 2017 che si contrappone alle RNN
evitando quindi la ricorrenza e utilizzando esclusivamente un meccanismo di
\textit{attention} per rappresentare i rapporti di dipendenza di input e output.
Si basa su di una struttura \textit{encoder-decoder} dove l'encoder fornisce al
decoder una rappresentazione dell'input ed in seguito il decoder fornisce una
frase in output \cite{vaswani2017attention}.

% BERT’s model architecture is a multi-layer bidirectional Transformer encoder
% based on the original implementation

Una delle più note architetture basata sul concetto di Transformer è
\textbf{BERT}, ovvero \textit{Bidirectional Encoder Representations from
Transformers}; si tratta di un transformer multi-layer bidirezionale, cioè in
grado di apprendere relazioni di dipendenza di un dato elemento dell'input sia
rispetto agli input precedenti che ai successivi, che si basa esclusivamente su
di moduli encoder. È stato introdotto con il fine di fornire un modello
pre-allenato semplicemente adattabile ad un vasto range di applicazioni tramite
\textit{fine-tuning}. Tuttavia anche gli approcci \textit{feature-based} basati
su BERT risultano efficaci \cite{devlin2018bert}.

In questo progetto BERT è stato utilizzato con quest'ultimo approccio
feature-based occupando effettivamente nel modello la stessa posizione dedicata
agli embedding per entrambi i campi nome e descrizione; il resto del modello è
pressochè invariato rispetto a quello precedentemente descritto, ovvero a
seguito di BERT troviamo fino a due layer LSTM bidirezionali (uno per il nome),
un GlobalMaxPooling1D e gli ultimi due layer Densi entrambi preceduti da un
dropout.

\subsubsection{Valutazione dell'errore}
% loss e metriche

Per quanto riguarda le performance in termini di errore la funzione di
\textit{loss} utilizzata in fase di training è il \textit{Root Mean Squared
Logaritmic Error (RMSLE)}, in quanto è la misura scelta dalla Kaggle challenge
per confrontare le performance dei vari partecipanti. Inoltre essa risulta
adeguata al problema considerando il vasto intervallo dei valori dei prezzi.

Di seguito la definizione e alcune osservazioni su di RMSLE e su un'ulteriore
metrica utilizzata, il \textit{Mean Absolute Error (MAE)},
che fornisce una più immediata comprensione rispetto al RMSLE. \\
Mean Absolute Error (MAE):
\begin{equation}
    \frac{1}{n} \sum_{i=1}^{n} | y_i - \hat{y_i} |
\end{equation}
\\
Mean Squared Logarithmic Error (MSLE): essendo calcolato a partire da un
rapporto riflette l'errore relativo e di conseguenza risulta efficace laddove i
valore assoluti presentano variazioni considerevoli.
\begin{equation}
    \frac{1}{n}
        \sum_{i=1}^{n}
            ( \log(y_i+1) - \log(\hat{y_i}+1) )^2
    =
    \frac{1}{n}
        \sum_{i=1}^{n}
            \log^2(\frac{y_i+1}{\hat{y_i}+1})
\end{equation}
\\
Root Mean Squared Logarithmic Error (RMSLE): la radice dell'MSLE.
\begin{equation}
    \sqrt{ 
        \frac{1}{n}
            \sum_{i=1}^{n}
                ( \log(y_i+1) - \log(\hat{y_i}+1) )^2
    }
\end{equation}

% dire da qualche parte qualche altra valutaione: e.g. la computazione richiesta
% dai vari approcci

\subsubsection{Valutazione dei costi computazionali}

Il numero dei parametri allenabili dei vari modelli sono stati annotati insieme
al tempo richiesto per l'esecuzione di 10 epoche di train sulla piattaforma
\textit{Google Colab} abilitando l'accelerazione hardware GPU.


\subsection{TODO}

adam, earlystopping, batchsize, learning rate.


\section{Results and Evaluation}
% The Results section is dedicated to presenting the actual results (i.e. measured and calculated quantities), not to discussing their meaning or interpretation. The results should be summarized using appropriate Tables and Figures (graphs or schematics). Every Figure and Table should have a legend that describes concisely what is contained or shown. Figure legends go below the figure, table legends above the table. Throughout the report, but especially in this section, pay attention to reporting numbers with an appropriate number of significant figures.
% \subsection{Risultati clean}
% \subsection{Risultati raw}

% \subsubsection{clean o raw}
% abbiamo valutato clean e raw; quali modelli beneficiano del cleanup, quali no?
% perchè? il deep learning impara meglio se il testo è raw?


% \begin{itemize}
%     \item bag of word based models: count vectorizer - tf/idf
%     \item embeddings: keras, glove pretrained. dire perchè non abbiamo provato
%     qualcosa tipo word2vec?
%     \item transformers con fine tuning
% \end{itemize}

\subsection{Modello finale}

Per quanto riguarda i layer RNN le tipologie di layer valutate sono: LSTM, GRU e
Bidirectional LSTM. Questi ultime, essendo in grado di catturare relazione sia
con parole precedenti che con e successive, si sono dimostrati più performanti e
sono stati utilizzati nel modello finale.

Sono stati inoltre valutati layer convoluzionali a monte dei layer RNN; tuttavia
non avendo migliorato considerevolevolmente le performance sono stati omessi nel
modello finale.
% todo numeri dei layers?

\subsection{Transformers}

% todo: l'ho scritto nel 03. andrà modificato un po'.
% necessario aggiungere performance
Non è stato purtroppo possibile per motivazioni computazionali allenare questo
modello sulla totalità del dataset ma è le limitazioni delle risorse di Google
Colab hanno reso obbligata la scelta di utilizzare una piccola porzione del
dataset per il training. Tuttavia i risultati sul validation sono
particolarmente buoni considerando la ridotta quantità di esempi di training a
disposizione rispetto ai modelli precedenti.

\section{Discussion}
% The discussion section aims at interpreting the results in light of the
% project's objectives. The most important goal of this section is to interpret
% the results so that the reader is informed of the insight or answers that the
% results provide. This section should also present an evaluation of the
% particular approach taken by the group. For example: Based on the results, how
% could the experimental procedure be improved? What additional, future work may
% be warranted? What recommendations can be drawn?
\subsection{Modello Finale}
\subsubsection{Parametri di training}

\paragraph{Numero di neuroni:} Il numero di neuroni del modello finale dei vari
layer è mostrato nelle figure \ref{fig:modeltemplate} e \ref{fig:kerasModel}
allo stesso modo del rate di dropout.

\paragraph{Optimizer:} Come optimizer è stato utilizzato Adam impostando il
learning rate a 1e-4, in quanto il training è risultato più stabile per la
maggior parte degli approcci ad eccezione dell'Embedding Keras che ha dimostrato
un comportamento migliore con il learning rate di default (1e-3).

\paragraph{Batchsize:} La batchsize è stata impostata a 256.

\paragraph{Regolarizzazione:} Oltre ai layer Dropout è stato testata la
regolarizzazione L2, tuttavia non ha portato a evidenti miglioramenti ed è
quindi stata omessa. Inoltre è stato impostato l'earlystopping a monitoraggio
dell'errore sul validation impostando pazienza a 3 epoche e il ripristino
automatico dei pesi più performanti.

\subsubsection{Layer aggiuntivi per il testo}

Per quanto concerne i layer aggiuntivi in coda agli Embedding sono state
valutate le seguenti tipologie di RNN: LSTM, GRU e Bidirectional LSTM. Queste
ultime, essendo in grado di catturare relazioni sia con parole precedenti che
con parole successive \cite{schuster1997bidirectional}, si sono dimostrate più
performanti e compaiono nel modello finale.

Sono stati inoltre valutati layer convoluzionali a precedere i layer RNN; tuttavia
non avendo migliorato considerevolmente le performance sono stati omessi nel
modello finale.

L'aggiunta di un layer GlobalMaxPooling1D a seguire le Bi\_LSTM (figura
\ref{fig:kerasModel}) si è dimostrata efficace sia per rendere la dimensionalità
compatibile con i successivi layer Densi che nell'effettuare un
\textit{downsampling} riducendo la dimensionalità insieme ai costi
computazionali.

\subsubsection{Dimensione Word Embedding}
La dimensione del word embedding Keras, ovvero quello allenato insieme al resto
del modello, è stata impostata a 50.

\subsection{Embedding pretrainati}
In tabella \ref{tab:restable} si nota come i modelli word embedding
pre-allenati (GloVe) richiedano un numero di epoche maggiore per convergere.

\subsection{Pulizia del testo}

Generalmente il preprocessing completo del testo non ha prodotto evidenti
miglioramenti sulle performance in termini di errore rispetto al preprocessing
limitato, nonostante nel caso degli embedding pre-trainati il preprossing
completo permette una maggiore copertura delle parole del vocabolario: da 32\%
a 38\% per GloVe6B e da 42\% a 52\% per GloVe840B.

Per quanto riguarda gli embedding ciò può essere dovuto al fatto che essendo in grado di catturare relazioni semantiche tra le parole non
necessitano di un alto grado di preprossesing che rimuovendo le stopwords, ad
esempio, potrebbe perfino complicare l'apprendimento del significato della
frase.

Risulta invece inaspettato il risultato ottenuto con il preprocessing totale
applicato a Bag of Word e Tf-Idf: l'errore è infatti leggermente più alto rispetto
a quello con preprocessing limitato. Si deduce che operazioni come
rimozione di stopwords e lemmatizzazione non hanno offerto un contributo
rilevante ma anzi un leggero peggioramento. 

Si nota invece una generale diminuzione del numero di parametri (figura
\ref{fig:resParam}) che in alcuni casi si traduce in una riduzione del tempo
richiesto da una singola epoca di train (figura \ref{fig:resEtime}).

\subsection{Performance}
Il modello più performante dal punto di vista dell'errore di regressione risulta
l'embedding Keras allenato con il resto del modello, evidenziato in verde nella
tabella \ref{tab:restable}. Seguono gli approcci Bag of Word e Tf-Idf;
quest'ultimo tuttavia non sembra migliorare i risulatati del più semplice BoW.

Dal punto di vista computazionale invece, la durata di una singola epoca di Bag
of Word è notevolmente minore rispetto a quella dell'embedding Keras allenato,
tuttavia considerando il numero di epoche necessarie alla convergenza ci
accorgiamo che i tempi di training risultano simili: 5000s per BoW e 5530s per
Keras. È necessario ricordare che però Keras è stato allenato con un learning
rate maggiore che per BoW risultava invece in una scarsa stabilità.

Sono invece gli Embedding GloVe pretrainati ad essere i più veloci nella singola
epoca di training, senza tener conto del modello di riferimento tenente conto
delle sole feature non testuali, avendo il numero minore di
parametri allenabili come mostrato in figura \ref{fig:resParam}. Anch'esse
tuttavia richiedono numerose epoche per convergere. Non risultano esserci
notevoli differenza ne di errori ne di costi computazionali tra i due embedding
pretrainati GloVe6B e GloVe840B.

\subsection{BERT}

Non è stato possibile per motivazioni computazionali allenare questo
modello sulla totalità del dataset ma è stata utilizzata porzione del
dataset per il training (circa 200 mila istanze) completando solo 5 epoche.
Tuttavia considerando queste limitazioni i risultati sembrano promettenti.

È stata una piacevole sorpresa il risultato ottenuto da Bag of Word che pur
essendo l'approccio più semplice testato ottiene ottimi risultati sia in termini
di errore che di richieste computazionali rappresentandone un buon compresso.

\section{Conclusions}
% Conclusions should summarize the central points made in the Discussion section,
% reinforcing for the reader the value and implications of the work. If the
% results were not definitive, specific future work that may be needed can be
% (briefly) described. The conclusions should never contain ``surprises''.
% Therefore, any conclusions should be based on observations and data already
% discussed. It is considered extremely bad form to introduce new data in the
% conclusions.

Riassumendo dai risultati ottenuti si evince che il modello Keras permette di
ottenere ottime performance in cambio di costi computazionali accettabili,
perciò questa tecnica risulta consigliata nel caso in cui si abbia a
disposizione una piattaforma almeno comparabile con Google Colab.

Bag of Words è stata una piacevole sorpresa infatti pur essendo l'approccio più
semplice testato ottiene ottimi risultati ed è probabilmente attuabile anche
da chi ha particolari limitazioni computazionali.

È stato inoltre evidenziato che il pre-processing del testo effettuato
generalmente non giova alle performance e sarebbe interessante valutare approcci
alternativi a quello utilizzato nel progetto.

Nel caso si abbiano a disposizione grandi risorse computazionali sarebbe utile
valutare approfonditamente l'approccio basato su BERT che potrebbe portare a
performance ottimali.

Un altro approccio che varrebbe la pena testare sarebbe l'utilizzo di un embedding
pre-trainato con GloVe come punto di partenza per poi allenarne ulteriormente i
pesi magari ampliando le dimensioni dell'embedding in caso si abbiano risorse sufficienti.


\bibliographystyle{IEEEtran}
\bibliography{Project}

\end{document}